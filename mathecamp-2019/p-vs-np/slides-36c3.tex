\documentclass[12pt,compress,ngerman,utf8,t]{beamer}
\usepackage[ngerman]{babel}
\usepackage{calc}
\usepackage{ragged2e,wasysym,multicol,mathtools,tikz,txfonts,ifthen}
\usepackage[all]{xy}
\usetikzlibrary{calc,shapes.callouts,shapes.arrows}
\usepackage[protrusion=true,expansion=true]{microtype}
\hypersetup{colorlinks=true}

\graphicspath{{images/}}

\title[P vs. NP]{$\varheartsuit$ P vs. NP $\varheartsuit$}
\author[Ingo Blechschmidt]{\scriptsize\textcolor{white}{
\vspace*{-1em} \\
\textbf{36th Chaos Communication Congress} \\
\emph{Questions are very much welcome! Please interrupt me mid-sentence.} \\
\medskip
Ingo Blechschmidt \\
University of Augsburg}}

%\usetheme{Warsaw}
\useinnertheme[shadow=false]{rounded}
\useoutertheme{split}
\usecolortheme{orchid}
\usecolortheme{whale}
\setbeamerfont{block title}{size={}}

\useinnertheme{rectangles}

\usecolortheme{seahorse}
\definecolor{mypurple}{RGB}{150,0,255}
\setbeamercolor{structure}{fg=mypurple}
\definecolor{myred}{RGB}{150,0,0}
\definecolor{darkred}{RGB}{240,0,0}
\setbeamercolor*{title}{bg=myred,fg=white}
\setbeamercolor*{titlelike}{bg=myred,fg=white}

\usefonttheme{serif}
\usepackage[T1]{fontenc}
\usepackage{libertine}

\renewcommand{\_}{\mathpunct{.}\,}
\renewcommand{\P}{\text{P}}
\newcommand{\NP}{\text{NP}}
\newcommand{\NPC}{\text{NP{-}C}}
\newcommand{\PSPACE}{\text{PSPACE}}
\newcommand{\PSPACEC}{\text{PSPACE{-}C}}
\newcommand{\EXP}{\text{EXP}}
\newcommand{\BB}{\mathbb{B}}
\newcommand{\M}{\mathcal{M}}
\newcommand{\R}{\mathrm{R}}
\newcommand{\NN}{\mathbb{N}}
\newcommand{\RR}{\mathbb{R}}
\newcommand{\Eff}{\mathrm{Eff}}
\newcommand{\TM}{\mathrm{TM}}
\newcommand{\STM}{\mathrm{STM}}
\newcommand{\RW}{\mathrm{RW}}
\newcommand{\lambdaC}{\lambda\mathrm{C}}
\newcommand{\PA}{\mathrm{PA}}
\newcommand{\goedel}[1]{\ulcorner #1 \urcorner}
\newcommand{\Prov}{\mathrm{Prov}}
\newcommand{\True}{\mathrm{True}}
\newcommand{\Con}{\mathrm{Con}}
\newcommand{\proves}{\vdash}
\newcommand{\defeq}{\vcentcolon=}

\newcommand{\pointthis}[4]{%
  \tikz[remember picture,baseline]{\node[anchor=base,inner sep=0,outer sep=0]%
    (#3) {#3};\node[overlay,rectangle callout,%
      callout relative pointer={(#1)},fill=blue!20] at
        ($(#3.north)+(#2)$) {#4};}}

\newcommand{\kasten}[1]{%
  \setlength{\fboxrule}{2pt}%
  \setlength{\fboxsep}{8pt}%
  {\usebeamercolor[fg]{item}\fbox{\usebeamercolor[fg]{normal text}\parbox{0.2cm}{#1}}}}%

\newcommand{\slogan}[1]{%
  \begin{center}%
    \setlength{\fboxrule}{2pt}%
    \setlength{\fboxsep}{8pt}%
    {\usebeamercolor[fg]{item}\fbox{\usebeamercolor[fg]{normal text}\parbox{0.88\textwidth}{#1}}}%
  \end{center}%
}

\newcommand{\code}[1]{%
  \begin{center}%
    \setlength{\fboxrule}{1pt}%
    \setlength{\fboxsep}{8pt}%
    {\fbox{\parbox{0.81\textwidth}{#1}}}%
  \end{center}%
}

\setbeamertemplate{navigation symbols}{}

\setbeamertemplate{title page}[default][colsep=-1bp,rounded=false,shadow=false]
\setbeamertemplate{frametitle}[default][colsep=-2bp,rounded=false,shadow=false,center]

\newcommand{\hil}[1]{{\usebeamercolor[fg]{item}{\textbf{#1}}}}
\newcommand{\bad}[1]{\textbf{\textcolor{darkred}{#1}}}
\setbeamertemplate{headline}{}
\setbeamertemplate{frametitle}{%
  \vskip0.6em%
  \leavevmode%
  \begin{beamercolorbox}[dp=1ex,center]{}%
      \usebeamercolor[fg]{item}{\textbf{\textsf{\Large \insertframetitle}}}
  \end{beamercolorbox}%
}

\setbeamertemplate{footline}{%
  \leavevmode%
  \hfill%
  \begin{beamercolorbox}[ht=2.25ex,dp=1ex,right]{}%
    \usebeamerfont{date in head/foot}
    \insertframenumber\,/\,\inserttotalframenumber\hspace*{1ex}
  \end{beamercolorbox}%
  \vskip0pt%
}

\newcommand{\backupstart}{
  \newcounter{framenumberpreappendix}
  \setcounter{framenumberpreappendix}{\value{framenumber}}
}
\newcommand{\backupend}{
  \addtocounter{framenumberpreappendix}{-\value{framenumber}}
  \addtocounter{framenumber}{\value{framenumberpreappendix}}
}

\newcommand{\portrait}[4]{\begin{column}{#3\textwidth}\centering\includegraphics[height=#4\textheight]{#1}\\{\scriptsize #2\par}\end{column}}

\setbeameroption{show notes}

% Taken from Todd Lehman (CC-BY-SA) at https://tex.stackexchange.com/a/44920/32372

\newcommand{\setisprime}[1]{
  % Sets \isprime based on #1.
  \ifnum#1=1 \gdef\isprime{0} \else \gdef\isprime{1} \fi
  \foreach \sip in {2, 3,5,...,#1} {
    \pgfmathparse{\sip*\sip>#1? 1:0}
    \ifthenelse{\pgfmathresult=1}{
      % Early-out if \sip^2 > #1.
      \breakforeach
    }{
      % Otherwise test if \sip divides #1.
      \pgfmathparse{Mod(#1,\sip)==0? 1:0}
      \ifthenelse{\pgfmathresult=1}{
        \gdef\isprime{0}
        \breakforeach
      }{}
    }
  }
}

\newcommand{\setxy}[1]{
  % Sets \x and \y to loction of cell #1.
  \pgfmathtruncatemacro{\x}{Mod(#1-1,\cols)}
  \pgfmathtruncatemacro{\y}{(#1-1) / \cols}
  \pgfmathtruncatemacro{\y}{\cols - 1 - \y}
  \pgfmathparse{2.5*(\x+.5)}\let\x\pgfmathresult
  \pgfmathparse{2.5*(\y+.5)}\let\y\pgfmathresult
}

\newcommand{\numlabel}[2]{
  % Draws label #2 at cell #1.
  \setxy{\n}
  \node[fill=none, text=black] at (\x,\y) {#2};
}

\newcommand{\drawpolygon}[2]{
  % Draws polygon with #2 vertexes at cell #1.
  \setxy{#1}
  \ifthenelse{#2>1}{ % Polygon must have at least 2 sides.
    \ifthenelse{#2<30}{ % Draw polygon if it has a small number of sides.
      \filldraw (\x,\y) +(90:1)
      \foreach \drawi in {1,...,#2} {-- +(\drawi/#2*360+90:1)} -- cycle;
    }{ % Else approximate with circle.
      \filldraw (\x,\y) circle(1);
    }
  }{}
}

\newcommand{\setpolygoncolor}[1]{
  % Sets color based on #1.
  \gdef\polycolor{black}
  \ifnum#1=2\gdef\polycolor{black!50!white}\fi
  \ifnum#1=3\gdef\polycolor{yellow!95!red}\fi
  \ifnum#1=5\gdef\polycolor{yellow!0!red}\fi
  \ifnum#1=7\gdef\polycolor{blue!75!green}\fi
  \ifnum#1=11\gdef\polycolor{blue!70!red}\fi
  \ifnum#1=13\gdef\polycolor{blue!40!red}\fi
  \ifnum#1=17\gdef\polycolor{green!50!blue}\fi
  \ifnum#1=19\gdef\polycolor{green!80!black}\fi
  \ifnum#1=23\gdef\polycolor{green!50!red}\fi
  \ifnum#1=29\gdef\polycolor{yellow!50!black}\fi
  \ifnum#1=31\gdef\polycolor{orange!50!black}\fi
  \ifnum#1=37\gdef\polycolor{red!50!black}\fi
  \ifnum#1=41\gdef\polycolor{purple!50!black}\fi
  \ifnum#1=43\gdef\polycolor{blue!50!black}\fi
  \ifnum#1=47\gdef\polycolor{green!50!black}\fi
  \ifnum#1=53\gdef\polycolor{white!50!black}\fi
  \ifnum#1=59\gdef\polycolor{white!50!black}\fi
  \ifnum#1=61\gdef\polycolor{white!50!black}\fi
  \ifnum#1=67\gdef\polycolor{white!50!black}\fi
}

\newcommand{\sieve}[2]{
  \def\cols{#1}
  \def\rows{#2}
  \begin{tikzpicture}[scale=.5,anchor=center]
  \pgfmathtruncatemacro{\nmax}{\rows * \cols}

  \foreach \n in {1,...,\nmax} {
    \begin{scope}[fill=gray, fill opacity=.05,
                  draw=gray, draw opacity=.10,
                  line width=4]
      \drawpolygon{\n}{\n}
    \end{scope}
    \setisprime{\n}
    \ifthenelse{\isprime=1}{
      \numlabel{\n}{\bf\n}
    }{
      \def\startintensity{.33}
      \def\incrintensity{.10}
      \def\intensity{\startintensity}

      \def\m{\n}
      \pgfmathtruncatemacro{\i}{\m / 2}

      % Divide \m by \i until \m is extinguished.
      % Increment \i each time it does not divide into \m.
      \whiledo{\m>1}{
        \setisprime{\i}
        \pgfmathparse{Mod(\m,\i)==0? 1:0}
        \ifthenelse{\pgfmathresult=1\and\isprime=1}{
          \setpolygoncolor{\i}
          \begin{scope}[fill=\polycolor, fill opacity=\intensity,
                        draw=\polycolor!85!black, draw opacity=\intensity,
                        line width=\intensity*1.5]
            \drawpolygon{\n}{\i}
          \end{scope}
          \pgfmathtruncatemacro{\m}{\m / \i}
          \pgfmathparse{\intensity + \incrintensity}\let\intensity\pgfmathresult
        }{
          \pgfmathtruncatemacro{\i}{\i - 1}
          \def\intensity{\startintensity}
        }
      }
      \begin{scope}[text=black, text opacity=.5]
        \numlabel{\n}{\scriptsize\n}
      \end{scope}
    }
  }

  \end{tikzpicture}
}


\begin{document}

\addtocounter{framenumber}{-1}

{\usebackgroundtemplate{\begin{minipage}{\paperwidth}\vspace*{0.3cm}\centering\scriptsize\sieve{25}{2}\\\vspace*{3.95cm}\includegraphics[width=\paperwidth]{sun3}\end{minipage}}
\begin{frame}[c]
  \centering

  \bigskip
  \bigskip
  \bigskip
  \Large

  \hil{$\varheartsuit$ P vs. NP $\varheartsuit$}
  \medskip

  \normalsize
  the biggest open question in computer science
  \bigskip
  
  \footnotesize
  \textit{-- an invitation --}
  \bigskip
  \bigskip
  \bigskip
  \bigskip
  \medskip

  \textcolor{black}{
    \textbf{36th Chaos Communication Congress} \\
    \emph{Questions are very much welcome! Please interrupt me mid-sentence.}
  }
  \bigskip
  \bigskip

  \tiny
  \textcolor{white}{
    Ingo Blechschmidt \\
    University of Augsburg
  }

  \par
\end{frame}}


% \begin{document}

\begin{frame}{The landscape of complexity classes}
  \justifying
  \textbf{Def.} An algorithm~$A$
  \hil{runs in polynomial time} if and only if there is some polynomial~$p$
  such that, for every input~$I$
  \[ \text{number of steps for~$A(I)$} \quad\leq\quad p(|I|), \] where~$|I|$ is
  the length of an encoding of~$I$ in bits.
  \medskip

  \begin{columns}
    \begin{column}{0.42\textwidth}
      \justifying
      \textbf{Def.} A problem is \hil{in P} if and only if there is
      is a decision algorithm which \hil{runs in polynomial time}.
      \medskip

      \textbf{Ex.} Primality testing, node reachability, \ldots
    \end{column}

    \begin{column}{0.57\textwidth}
      \justifying
      \textbf{Def.} A problem is \hil{in NP} if and only if
      there is an algorithm which verifies \hil{wannabe certificates for a positive
      answer} in polynomial time.
      \medskip

      \textbf{Ex.} 3SAT, Sudoku, TSP, graph coloring, proof search, \ldots
    \end{column}
  \end{columns}
  \medskip

  \textbf{Prop.} Every~$\P$-problem is also in~$\NP$: $\P \subseteq \NP$.
\end{frame}

\begin{frame}{Further complexity classes}
  \vspace*{-1.6em}\justifying
  \[ \P \subseteq
  \begin{array}[t]{@{}c@{}}\NP\\\rotatebox{90}{$\subseteq$}\\\NPC\end{array} \subseteq
  \begin{array}[t]{@{}c@{}}\PSPACE\\\rotatebox{90}{$\subseteq$}\\\PSPACEC\end{array} \subseteq
  \EXP \]

  A problem~$T$ is in \ldots
  \begin{itemize}\justifying
    \item \hil{$\P$} iff there is a \hil{polynomial-time} decision algorithm.
    \item \hil{$\NPC$} iff it is in~$\NP$ and if
    every~$\NP$-problem is \hil{reducible} to~$T$ in polynomial time.
    \item \hil{$\NP$} iff there is a polynomial-time algorithm which verifies wannabe
    certificates for a positive answer.
    \item \mbox{\hil{$\PSPACE$} iff there is a \hil{polynomial-space} decision
    algorithm.}
    \item \hil{$\PSPACEC$} iff it is in~$\PSPACE$ and if
    every~$\PSPACE$-problem is reducible to~$T$ in polynomial time.
    \item \hil{$\EXP$} iff there is an \hil{exponential-time} decision algorithm.
  \end{itemize}

  \textbf{Rem.} $\P \neq \EXP$, hence $\P \neq \NP$ or $\NP \neq \PSPACE$ or
  $\PSPACE \neq \EXP$.
\end{frame}

\begin{frame}{The relativization barrier}
  \justifying
  Let~$A$ be a problem.
  \bigskip

  \textbf{Def.} A problem is in~$\P^A$ iff there is a polynomial-time decision
  algorithm which may ask an \hil{oracle for~$A$}.
  \bigskip

  \textbf{Def.} A problem is in~$\NP^A$ iff there is a polynomial-time
  algorithm which verifies wannabe certificates for a positive answer
  algorithm which may ask an \hil{oracle for~$A$}. XXX
  \bigskip

  \textbf{Prop.} $\P^A \subseteq \NP^A \subseteq \PSPACE^A$.
  \bigskip

  \textbf{Prop.} If~$A$ is in~$\NPC$, then~$\NP \subseteq \P^A$.
  \bigskip

  \textbf{Thm.} For some~$A$, $\P^A = \NP^A$; and for some other~$A$, $\P^A \neq
  \NP^A$.
\end{frame}

\end{document}
